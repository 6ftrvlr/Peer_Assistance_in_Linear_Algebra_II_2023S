\documentclass[a4paper]{article}
\usepackage[left=0.6in,right=0.6in,top=0.8in,bottom=1in]{geometry}
\usepackage{amscd,amssymb,amsfonts,amsbsy,amsmath,verbatim,color, mathrsfs}
\usepackage{tikz}
\usepackage{pgfplots}
\usepackage{tikz,tikz-3dplot}
\usepackage{metalogo}
\usepackage{xeCJK}  %务必使用XeLaTeX编译!!!
\usepackage{fontspec}
\usepackage{lastpage}
\usepackage{fancyhdr}
\usepackage{pgfplots}
\usepackage{tikz}
\usepackage{fancyhdr}
\usepackage{graphicx}
\usepackage{makecell}
\usepackage{diagbox}
\usepackage{indentfirst}
\usepackage{setspace}
\usepackage{bm}
\usepackage{enumerate}
\usepackage{graphicx}
\usepackage{color}
\usepackage{colortbl}
%\usepackage{background}
\usepackage{lipsum}
\usepackage{wrapfig}
\usepackage{picinpar}
\usepackage{xpatch}
\usepackage{yhmath}
\usepackage[colorlinks,linkcolor=black,anchorcolor=blue,citecolor=green]{hyperref}
\newcommand\bi{(\item)}
\newcommand\ee{\mathrm{e}}
\newcommand\dd{\mathrm{d}}
\newcommand\ii{\mathrm{i}}
\newcommand\CC{\mathrm{C}}
\newcommand\bsy{\boldsymbol}
\newcommand\CCC{\mathbb{C}}
\newcommand\sqm{n \times n}

\newtheorem{exa}{例题}[subsection]
\newtheorem{exec}{习题}[section]
\newtheorem{thm}{定理}[subsection]

% 应同学的请求,公布LaTeX源代码.LaTeX真好玩.
% 不过玩得不是很明白啊,前面一坨调包,看着个新包就调,
% 调得越来越长,以至于我都不知道有些包有些代码是干啥的了.
% 另外,CJK字体设置也是我的知识盲区,
% 我拿那个什么CJK什么family那句命令加载个字体全篇都跟着动.
% 所以PDF文档里面标题字体不太一样,那是我拿Adobe Acrobat改的.
% 对了,multirow我也不太会用.有道题的题文,直接编译文字是断开的,
% 文档里面是连着的,那也是我拿Adobe Acrobat改的.
% 另外,码风不太好,草盛豆苗稀.
% 如果同学会用,还请不吝赐教.
% 不过,看来我还是更会玩Acrobat呢.

\begin{document}
\begin{spacing}{1.5}

\begin{center}
    \begin{Large}
    
    {《高等代数》课后辅导 :期中考前练习}
    \end{Large}
\end{center}

\everymath{\displaystyle}
\begin{center}
    
    \textit{试题共2页,6道大题.建议完成时间为110min.}
    
    
\end{center}

%%%%%%%%%%%%%%%%%%%%
%%%%%    %%%
%   %   %  %
%   %  %   %
%   %      %
%%%%%  %%%%%
%%%%%%%%%%%%%%%%%%%%
\vspace{5mm}
\begin{table}[!ht]
	\renewcommand\arraystretch{2}
	\begin{tabular}{m{1cm}<{\centering}m{12cm}m{2cm}m{1.5cm}<{\centering}}
 \Large\cellcolor{black} \textcolor{white}{\textsf{01}}  & \large \cellcolor{lightgray}\textsf{互素多项式} & &\cellcolor{lightgray} \large{12\%} \\
\end{tabular}
\end{table}设$f(x)=x^3-x+2,g(x)=x^2+1\in\mathbb{Q}[x]$.验证$(f,g)=1$,并求$u(x),v(x)\in\mathbb{Q}[x]$,使得$u(x)f(x)+v(x)g(x)=1$.
\vspace{20mm}





%%%%%%%%%%%%%%%%%%%%
%%%%%  %%%%%
%   %      %
%   %  %%%%%
%   %  %
%%%%%  %%%%%
%%%%%%%%%%%%%%%%%%%%
\vspace{5mm}
\begin{table}[!ht]
	\renewcommand\arraystretch{2}
	\begin{tabular}{m{1cm}<{\centering}m{12cm}m{2cm}m{1.5cm}<{\centering}}
 \Large\cellcolor{black} \textcolor{white}{\textsf{02}}  & \large \cellcolor{lightgray}\textsf{矩阵的不变因子} & &\cellcolor{lightgray} \large{20\%} \\
\end{tabular}
\end{table}已知矩阵$\bsy{A}\in\mathbb{C}^{8\times8}$的不变因子除却$6$个1以外还有多项式$(\lambda - 2)(\lambda ^2+4)^s$与$(\lambda - 2)^t(\lambda ^2+4)$,其中$s$与$t$都是正整数,且$t>1$.请求解下面内容,并给出完整步骤或原理说明:
\begin{spacing}{1.7}
\begin{tabular}{p{5cm}p{5cm}p{5cm}}
%   \toprule
    \textbf{i.}$s,t$的值;&
    \textbf{ii.}$\bsy{A}$的所有初等因子;&
    \textbf{iii.}$\bsy{A}$的Jordan标准形;\\
    % \midrule
    \textbf{iv.}$\bsy{A}$的最小多项式;&
    \textbf{v.}$\bsy{B}=\bsy{A}^{60}$的最小多项式;&\\
    \textbf{vi.}若矩阵$\bsy{A}$可看作实数矩&阵,$\bsy{A}$的有理标准形.&\\
    % \bottomrule
\end{tabular}
\end{spacing}
\vspace{20mm}







%%%%%%%%%%%%%%%%%%%%
%%%%%  %%%%%
%   %      %
%   %  %%%%%
%   %      %
%%%%%  %%%%%
%%%%%%%%%%%%%%%%%%%%
\vspace{5mm}
\begin{table}[!ht]
	\renewcommand\arraystretch{2}
	\begin{tabular}{m{1cm}<{\centering}m{12cm}m{2cm}m{1.5cm}<{\centering}}
 \Large\cellcolor{black} \textcolor{white}{\textsf{03}}  & \large \cellcolor{lightgray}\textsf{线性变换} & &\cellcolor{lightgray} \large{24\%} \\
\end{tabular}
\end{table}

设$V$为一复数域上的三维线性空间,$\mathscr{A}$为该线性空间上的一个线性变换,且$\mathscr{A}$在基$\bsy{e}_1,\bsy{e}_2,\bsy{e}_3$下的矩阵为$\bsy{A} = \begin{pmatrix}4&-3&4\\3&-2&4\\8&-8&9\end{pmatrix}.$
\begin{enumerate}
        \item[\textbf{i.}] 试以$\bsy{e}_1,\bsy{e}_2,\bsy{e}_3$的线性组合表示出一组基$\bsy{f}_1,\bsy{f}_2,\bsy{f}_3$,使得$\mathscr{A}$在基$\bsy{f}_1,\bsy{f}_2,\bsy{f}_3$下的矩阵为Jordan标准型;
        \item[\textbf{ii.}]若该线性空间上的一个线性变换$\mathscr{F}$满足$\mathscr{F}^2+9\mathscr{E}=10\mathscr{F}$(其中$\mathscr{E}$为恒等变换),则该线性变换在某组基下的矩阵是否可能为$\bsy{A}$?请说明理由.
        \item[\textbf{iii.}]是否存在一个线性变换$\mathscr{B}$满足$\mathscr{B}^2=\mathscr{A}$?若存在,请求出$\mathscr{B}$在基$\bsy{e}_1,\bsy{e}_2,\bsy{e}_3$下的矩阵;若不存在,请证明之;
        \item[\textbf{iv.}]记集合$C$为$V$上所有满足$\mathscr{AC}=\mathscr{CA}$(即可交换性)的线性变换组成的集合,试证明集合$C$(在线性变换之加法与数乘运算的意义下)构成复数域上的线性空间,并计算它的维数.
    \end{enumerate}






\vspace{15mm}






%%%%%%%%%%%%%%%%%%%%
%%%%%  %   %
%   %  %   %
%   %  %%%%%
%   %      %
%%%%%      %
%%%%%%%%%%%%%%%%%%%%
\begin{table}[!ht]
	\renewcommand\arraystretch{2}
	\begin{tabular}{m{1cm}<{\centering}m{12cm}m{2cm}m{1.5cm}<{\centering}}
 \Large\cellcolor{black} \textcolor{white}{\textsf{04}}  & \large \cellcolor{lightgray}\textsf{多项式的整除问题} & &\cellcolor{lightgray} \large{14\%} \\
\end{tabular}
\end{table}
证明:多项式$f(x)=a_nx^n+a_{n-1}x^{n-1}+\cdots+a_1x+a_0$能被$(x-1)^k$整除的充要条件为
$$\left( \begin{matrix}
	1&		1&		1&		\cdots&		1\\
	1&		2&		3&		\cdots&		n\\
	1&		4&		9&		\cdots&		n^2\\
	\vdots&		\vdots&		\vdots&		\ddots&		\vdots\\
	1&		2^k&		3^k&		\cdots&		n^k\\
\end{matrix} \right) \left( \begin{matrix}
	a_0\\
	a_1\\
	a_2\\
	\vdots\\
	a_n\\
\end{matrix} \right) =\left( \begin{matrix}
	0\\
	0\\
	0\\
	\vdots\\
	0\\
\end{matrix} \right) .$$


\vspace{5mm}











%%%%%%%%%%%%%%%%%%%%
%%%%%  %%%%%
%   %  %   
%   %  %%%%%
%   %      %
%%%%%  %%%%%
%%%%%%%%%%%%%%%%%%%%
\begin{table}[!ht]
	\renewcommand\arraystretch{2}
	\begin{tabular}{m{1cm}<{\centering}m{12cm}m{2cm}m{1.5cm}<{\centering}}
 \Large\cellcolor{black} \textcolor{white}{\textsf{05}}  & \large \cellcolor{lightgray}\textsf{中国剩余定理} & &\cellcolor{lightgray} \large{16\%} \\
\end{tabular}
\end{table}

\begin{enumerate}
\item[\textbf{I.}] 若$p_k(x)\in \mathbb{P} [x](k=1,2,\cdots,s)$是一组两两互素的非零次多项式,~$f_k(x)\in \mathbb{P} [x](k=1,2,\cdots,s)$为一组不全为零的多项式,证明下面两个命题:
    \begin{enumerate}
        \item[\textbf{i.}] $\exists h_k(x)\in \mathbb{P} [x](k=1,2,\cdots,s)$,使得$$ p_i(x)|h_k(x)-\delta_{ik} ,\forall i=1,2,\cdots,s,$$
其中$\delta_{ik}$为Kronecker符号,当$i=k$时取1,否则取0;

        \item[\textbf{ii.}]$\exists F(x)\in\mathbb{P}[x]$,使得$$p_i(x)|F(x)-f_i(x),\forall i = 1,2,\cdots,s.$$
    \end{enumerate}
\item[\textbf{II.}] 对于问题\textsf{01}中的多项式$f(x)$与$g(x)$,求一个次数不超过4的多项式$h(x)\in\mathbb{Q}[x]$,  使得
$$f(x)|h(x)+4x,g(x)|h(x)-4.$$
\end{enumerate}
\vspace{5mm}











%%%%%%%%%%%%%%%%%%%%
%%%%%  %%%%%
%   %  %   
%   %  %%%%%
%   %  %   %
%%%%%  %%%%%
%%%%%%%%%%%%%%%%%%%%

\begin{table}[!ht]
	\renewcommand\arraystretch{2}
	\begin{tabular}{m{1cm}<{\centering}m{12cm}m{2cm}m{1.5cm}<{\centering}}
 \Large\cellcolor{black} \textcolor{white}{\textsf{06}}  & \large \cellcolor{lightgray}\textsf{矩阵指数} & &\cellcolor{lightgray} \large{14\%} \\
\end{tabular}
\end{table}

设$\bsy{A}$为$n$阶方阵且$\bsy{A}^k=\bsy{O}$.定义矩阵$$\bsy{X}=\bsy{E}+\bsy{A}+\frac{\bsy{A}^2}{2!}+\cdots+\frac{\bsy{A}^{k-1}}{(k-1)!},$$其中矩阵$\bsy{E}$为单位矩阵.证明:

\begin{enumerate}
    \item[\textbf{i.}] 矩阵$\bsy{X}$可逆;
    
    \item[\textbf{ii.}] 矩阵$\bsy{X}$与矩阵$\bsy{E+A}$相似.
\end{enumerate}


% % 运用矩阵Jordan标准形理论探索\textsf{不可逆}复数矩阵存在“平方根”之充要条件.所谓复数矩阵$\bsy B$存在“平方根”,即$\exists \bsy R \in \CCC^{\sqm},\bsy R^2= \bsy B.$
% \vspace{15mm}




\end{spacing}





\end{document}
